\documentclass[12pt, letterpaper, a4paper]{article}
\usepackage{graphicx}
\usepackage{textcomp}
\usepackage[hungarian]{babel}
\usepackage[T1]{fontenc}
\usepackage[utf8]{inputenc}
\usepackage{caption}
\usepackage{subcaption}
\usepackage{csquotes}
\graphicspath{{./Pictures/}}
\usepackage[
    style=ieee,
  ]{biblatex}
\addbibresource{onlab.bib}

\usepackage{listings}
\usepackage[svgnames]{xcolor}
\usepackage{tikz}
\usetikzlibrary{shapes.geometric, arrows, calc}

\usepackage{float}
\usepackage{amsmath}
\usepackage{tabularx}
\usepackage{url}

\definecolor{dkgreen}{rgb}{0,0.6,0}
\definecolor{gray}{rgb}{0.5,0.5,0.5}
\definecolor{mauve}{rgb}{0.58,0,0.82}


\lstset{frame=none,
  language=Bash,
  aboveskip=3mm,
  belowskip=3mm,
  showstringspaces=false,
  columns=flexible,
  basicstyle={\small\ttfamily},
  numbers=none,
  numberstyle=\tiny\color{gray},
  keywordstyle=\color{blue},
  commentstyle=\color{dkgreen},
  stringstyle=\color{mauve},
  breaklines=true,
  breakatwhitespace=true,
  tabsize=2
}

\lstdefinestyle{cstyle}{
  language=C,
  basicstyle=\footnotesize\ttfamily,  % smaller fixed-width font
  keywordstyle=\color{RoyalBlue},     % keywords like for, if, while
  commentstyle=\color{DarkGreen}\itshape,  % italic green comments
  stringstyle=\color{orange!80!black},     % orange strings
  identifierstyle=\color{black},      % variable/function names
  numberstyle=\tiny\color{gray},
  numbers=left,
  numbersep=8pt,
  frame=none,
  showspaces=false,
  showstringspaces=false,
  showtabs=false,
  tabsize=2,
  captionpos=b,
  breaklines=true,
  breakatwhitespace=true,
  aboveskip=0.5em,
  belowskip=0.5em,
  lineskip=-1pt,
  morekeywords={uint8_t, memcpy, memset}, % highlight C stdlib/uint types
  morekeywords=[2]{compare,merge,sort,media_filter_scalar},
  keywordstyle=[2]\color{purple}
}

\lstdefinestyle{pythonstyle}{
  language=Python,
  basicstyle=\ttfamily\footnotesize,
  keywordstyle=\color{RoyalBlue}\bfseries,
  commentstyle=\color{DarkGreen}\itshape,
  stringstyle=\color{orange!80!black},
  numberstyle=\tiny\color{gray},
  numbers=left,
  numbersep=8pt,
  frame=none,
  showstringspaces=false,
  tabsize=4,
  breaklines=true,
  breakatwhitespace=true,
  captionpos=b,
}

\tikzstyle{box} = [rectangle, rounded corners, minimum width=3cm, minimum height=0.5cm,text centered, draw=black, fill=orange!30]
\tikzstyle{arrow} = [thick,->,>=stealth]


\title{\textcolor{RoyalBlue}{\textbf{Audio tömörítő egység megvalósítása FPGA-val}}}
\author{\textcolor{RoyalBlue}{Nyiri Levente}}
\date{2025 Október}
\begin{document}
\begin{figure}
  \centering
  \includegraphics[width=\textwidth]{BME.png}
\end{figure}
\selectlanguage{hungarian}
\maketitle

\newpage

\section{Szabvány}

Tanulmányoztam az MPEG-1 Audio és az MPEG-2 Advanced Audio Coding (AAC) szabványokat, úgy döntöttem, hogy az MPEG-1 Audio-t fogom implementálni.

\section{LAME}

Kiindulásnak a LAME (Lame Aint an MP3 Encoder) nyílt forráskódú MP3 tömörítő szoftver forráskódját tanulmányoztam.
A weboldalukról töltöttem le, fordítottam és egy példa .wav file-on kipróbáltam.
%TODO még írni kicsit a LAME-ről


\section{Encoder}

Az encoder blokkvázlata\cite{mp3-iso} az \ref{fig:encoder-diagram}. árbán látható.
Nem szabványosított az algoritmus, én a LAME source code-jából fogok kiindulni a tervezésnél.
A kimeneti bitsream-nek meg kell felelnie az International Standard-nak.

\begin{figure}[H]
  \centering
  \includegraphics[width=\textwidth]{encoder-diagram.png}
  \caption{Encoder blokkvázlat}
  \label{fig:encoder-diagram}
\end{figure}

\subsection{Mapping}


\subsection{Frame}

Egy MP3 file kisebb részegységekre van osztva, ezeket frame-eknek hívjuk. Minden frame 1152 audio mintát tartalmaz. Egy frame továbbá szét van választva 2 granule-ra, mindkettőre 576 minta jut.

Egy frame méretét byteban a(z) \ref{eq:frame-size}. egyenlet írja le.   

\begin{equation}
    \label{eq:frame-size}
    \text{frame size} = \frac{144 \cdot \text{bitrate}}{f_s} + \text{Padding}
\end{equation}
Egy frame felépítése a(z) \ref{tab:frame-header}. táblázatban \cite{raissi-mp3} látható.

\begin{table}[h]
\centering
\begin{tabular}{|c|c|c|c|c|}
  \hline
  \texttt{Header} & \texttt{CRC} & \texttt{Side\_Information} & \texttt{Main\_Data} & \texttt{Ancillary\_Data} \\
  \hline
\end{tabular}
\caption{Az MPEG frame fő mezői}
\label{tab:frame-header}
\end{table}

\subsubsection{Header}

A header tartalmazza a szinkronizációs biteket és egyéb információkat a frame-ről, felépítését mutatja a(z) \ref{fig:frame-header}. ábra \cite{raissi-mp3}.

\begin{figure}[H]
  \centering
  \includegraphics[width=0.2\textwidth]{frame-header.png}
  \caption{Frame header}
  \label{fig:frame-header}
\end{figure}

\begin{description}
  \item[Sync (12 bit)] Szerepe a szinkronizálás, mind a 12 bitnek 1-esnek kell lennie: \texttt{sync = 12'b1111\_1111\_1111}
  \item[ID (1 bit)] Az MPEG verziót határozza meg (MPEG-1 vagy MPEG-2)
  \item[Layer (2 bit)] Layer I, II vagy III 
  \item[Protection bit (1 bit)] Meghatározza, használunk-e CRC-t
  \item[Bitrate (4 bit)] A bitrate beállítása
  \item[Frequency (2 bit)] A mintavételi frekvencia meghatározása
  \item[Padding bit (1 bit)] Néhány frame-nek szüksége van rá, hogy a bitrate pontos legyen
  \item[Private bit (1 bit)] Applikáció-specifikus trigger
  \item[Mode (2 bit)] Csatornamód 
  \item[Mode extension (2 bit)] Csak joint stereo esetén használatos, további specifikáció
  \item[Copyright bit (1 bit)] Jelzi, ha a tartalom szerzői jogi védelem alatt áll
  \item[Home bit (1 bit)] A frame az eredeti adathordozón található-e
  \item[Emphasis (2 bit)] A dekódernek szükséges-e de-emphasist alkalmaznia zajcsökkentés után
\end{description}

\subsubsection{Side information}

A side information további információt tartalmaz arra vonatkozóan, hogy hogyan kell dekódolni a frame-et. A felépítését a(z) \ref{tab:side-info}. táblázat mutatja.


\begin{table}[h]
\centering
\footnotesize
\begin{tabular}{|c|c|c|c|c|}
  \hline
  \texttt{main\_data\_begin} & \texttt{private\_bits} & \texttt{scfsi} & \texttt{Side\_info gr. 0} & \texttt{Side\_info gr. 1} \\
  \hline
\end{tabular}
\caption{A side information mezői}
\label{tab:side-info}
\end{table}

\begin{description}
  \item [main\_data\_begin (9 bit)] Layer III-nál bit reservoir használatával egy adott frame main data helyén szabadon maradt helyet másik frame-ek is használhatják. Ez a mező azt a negatív offsetet adja meg, amennyivel korábban kezdődik egy frame-nek a main data-ja a sync word-höz képest.
  \item [private\_bits (5 bit)] Szabad felhasználás 
  \item [scfsi (4-4 bit)] ScaleFactor Selection Information, azt határozza meg, hogy az egyes scalefactor-ok mindkét granule-ra használhatóak-e vagy külön kell mindkettőre küldeni. Ha elég csak egyet küldeni, azzal biteket nyerünk, amit Huffman kódoláshoz lehet használni. Layer III-nál a scalefactorok 4 csoportba vannak osztva a(z) \ref{tab:scalefactor-bands}. táblázat szerint.

    Mindkét csatornára vonatkozóan, ha egy bit 1-esbe van, akkor az adott csoporthoz tartozó scalefactorokat a frameben lévő második granule újra fogja használni.

    Ha short window-t használunk (block\_type == 2) bármely granule-ban, akkor a scalefactorokat mindig külön küldjük mindkét csatornára.

\begin{table}[h]
\centering
\footnotesize
\begin{tabular}{|c|c|}
  \hline
  \textbf{group} & \textbf{scalefactor bands} \\
  \hline
  0 & 0,1,2,3,4,5 \\
  \hline
  1 & 6,7,8,9,10 \\
  \hline
  2 & 11,12,13,14,15 \\
  \hline
  3 & 16,17,18,19,20 \\
  \hline
\end{tabular}
\caption{Scalefactor csoportok}
\label{tab:scalefactor-bands}
\end{table}

\end{description}

A granule-okhoz tartozó \texttt{side\_info} még további mezőkből áll, ezt mutatja a(z) \ref{tab:granule-side-info}. táblázat.
\begin{table}[h]
\centering
\footnotesize
\begin{tabular}{|l|l|l|l|}
  \hline
  \texttt{part2\_3\_length} & \texttt{big\_values} & \texttt{global\_gain} & \texttt{scalefac\_compress} \\
  \hline
  \texttt{windows\_switching\_flag} & \texttt{block\_type} & \texttt{mixed\_block\_flag} & \texttt{table\_select} \\
  \hline
  \texttt{subblock\_gain} & \texttt{region0\_count} & \texttt{region1\_count} & \texttt{preflag} \\
  \hline
  \texttt{scalefac\_scale} & \texttt{count1table\_select} & \multicolumn{2}{c}{} \\
  \cline{1-2}
\end{tabular}
\caption{Granule side information mezői}
\label{tab:granule-side-info}
\end{table}

\begin{description}
  \item [par2\_3\_length (12-12 bit)] Megmondja, hogy a main data részében a frame-nek hány bit van allokálva scalefactor-oknak (part2) és Huffman kódolt adatnak (part3).
  \item [big\_values (9-9 bit)] Az egyes granule-ok spektrális komponensei más Huffman kód táblázatokkal vannak kódolva. A teljes spektrum 0-tól a Nyquist frekvenciáig több részre van bontva, és ezek a részek máshogy vannak kódolva. A partícionálás a maximális kvantált értékek alapján történik. A magasabb frekvenciájú komponenseknek várhatóan alacsonyabb amplitúdójuk van, vagy nem is kell őket kódolni. Megszámoljuk, hogy minden frekvencián összesen hány 0 érték van, ezeknek a számát az \texttt{rzero}-ban tároljuk. 
    A \texttt{count1} mezőben 4-esével vannak értékek, és az abszolútértéke nem haladja meg az 1-et (az egyes értékek -1, 0 vagy 1 lehetnek). A többi érték a \texttt{big\_values} mezőben van, ezen belül is 3 részre osztva. Az abszolút maximum értéke ennek a régiónak 8191.
    
    A big\_values mező felosztása a(z) \ref{fig:big-values}. ábrán látható.
  
\begin{figure}[H]
  \centering
  \includegraphics[width=0.8\textwidth]{big-values.png}
  \caption{big\_values felosztása}
  \label{fig:big-values}
\end{figure}

  \item [global\_gain (8-8 bit)] Kvantálás lépésköz.
  \item [scalefac\_compress (4-4 bit)] Hány bitet használjon a scalefactorok átviteléhez. Hogy az értéke alapján az első és második scalefactor group hány bitet kap az a(z) \ref{tab:scalefac-compress}. táblázatban látható.
\begin{table}[h]
\centering
\footnotesize
\begin{tabular}{|c|c|c|}
  \hline
  \textbf{scalefac\_compress} & \textbf{slen1} & \textbf{slen2} \\
  \hline
  0 & 0 & 0 \\
  \hline
  1 & 0 & 1 \\
  \hline
  2 & 0 & 2 \\
  \hline
  3 & 0 & 3 \\
  \hline
  4 & 3 & 0 \\
  \hline
  5 & 1 & 1 \\
  \hline
  6 & 1 & 2 \\
  \hline
  7 & 1 & 3 \\
  \hline
  8 & 2 & 1 \\
  \hline
  9 & 2 & 2 \\
  \hline
  10 & 2 & 3 \\
  \hline
  11 & 3 & 1 \\
  \hline
  12 & 3 & 2 \\
  \hline
  13 & 3 & 3 \\
  \hline
  14 & 4 & 2 \\
  \hline
  15 & 4 & 3 \\
  \hline
\end{tabular}
\caption{Scalefac compress értékek}
\label{tab:scalefac-compress}
\end{table}

  \item [windows\_switching\_flag (1-1 bit)] Megmutatja, ha a normálon kívül másmilyen ablaktípus van használatban.
  \item [block\_type (2-2 bit)] Ha nem normál ablakot használunk, mutatja, hogy milyet. A lehetőségek a \ref{tab:block-type}. táblázatban vannak.
\begin{table}[h]
\centering
\footnotesize
\begin{tabular}{|c|c|}
  \hline
  \textbf{block\_type} & \textbf{window type} \\
  \hline
  00 & forbidden \\
  \hline
  01 & start \\
  \hline
  10 & 3 short windows \\
  \hline
  11 & end \\
  \hline
\end{tabular}
\caption{Block type értékek}
\label{tab:block-type}
\end{table}

  \item [mixed\_blockflag (1-1 bit)] Akkor használható, ha  a windows\_switching\_flag be van állítva. Azt jelzi, hogy más ablaktípust használunk alacsonyabb és magasabb frekvenciákon. Az alsó 2 sávban normál, a maradék 30-ban pedig a block\_type-ban megadottal.

  \item [table\_select (10-10 bit)] Megadja milyen Huffman kódolást használjunk.
  \item [subblock\_gain (9-9 bit)] Ha windows\_switching\_flag set és block\_type == 10, akkor a gain offsetet mutatja a global gain-től.
  \item [region\_address1 (4-4 bit) region\_address2 (3-3 bit)] A spektrumot további részekre osztjuk a Huffman kódoláshoz, ezek a változók ezeknek a régióknak a kezdőcímét tartalmazzák.

  \item [preflag (1-1 bit)] További nagyfrekvenciás erősítése a kvantált mintáknak. Ha set, akkor értékeket ad hozzá a scalefactorokhoz.
  \item [scaleflac\_scale (1-1 bit)] A scalefactorok logaritmikusan kvantáltak a(z) \ref{tab:scalefac-scale}. táblázat szerint.
\begin{table}[h]
\centering
\footnotesize
\begin{tabular}{|c|c|}
  \hline
  \textbf{scalfac\_scale} & \textbf{step size} \\
  \hline
  0 & $\sqrt{2}$ \\
  \hline
  1 & 2 \\
  \hline
\end{tabular}
\caption{Scalefac scale értékek}
\label{tab:scalefac-scale}
\end{table}

  \item [count1table\_select (1-1 bit)] Huffman kódolást választ a count1 mezőben lévő értékekhez.
\end{description}

\subsubsection{Main Data}

Scalefactorokból, Huffman kódolt bitekből és ancillary data-ból áll. Az elrendezés a(z) \ref{fig:main-data-granules}. ábrán látható.

\begin{figure}[H]
  \centering
  \includegraphics[width=\textwidth]{main-data-granule.png}
  \caption{Main data, granule-ok és channelek elrendezése}
  \label{fig:main-data-granules}
\end{figure}

\begin{description}

  \item [Scale factors] A céljuk, hogy redukálják a kvantálási zajt. Ha a minták egy adott scalefactor band-ben megfelelően vannak scale-elve, akkor a kvantálási zaj teljesen ki lesz maszkolva.
  \item [Huffman code bits] Itt találhatóak a Huffman kódolt bitek.
  \item [Ancillary data] Opcionális, felhasználó definiálhatja.

\end{description}



\subsection{Megkötések}

Mivel egy áltanános, a header-ben megadható minden opciót kiszolgáló megvalósítás túl sok időbe telne, ezért bizonyos megkötésekkel élni fogunk.

A megkötéseket a(z) \ref{tab:bit-constraints}. táblázat mutatja.

\begin{table}[h]
\centering
\footnotesize
\begin{tabular}{|c|c|p{6cm}|}
  \hline
  \textbf{Mező Neve} & \textbf{Bit Érték} & \textbf{Jelentés} \\
  \hline
   ID & 1 & MPEG-1 \\
  \hline
  Layer & 01 & Layer III \\
  \hline
  Protection bit & 1 & Használunk CRC-t \\
  \hline
  Bitrate & 1110 & 320 Kbps \\
  \hline
  Frequency & 00 & 44.1 kHz mintavételi frekvencia \\
  \hline
  Mode & 00 & Stereo \\
  \hline
  Emphasis & 00 & Nem kell de-emphasis dekódolásnál, nem használunk noise suppression-t \\
  \hline
\end{tabular}
\caption{Megkötések a bitekre}
\label{tab:bit-constraints}
\end{table}


\subsection{Analysis Polyphase Filterbank}

\begin{figure}[H]
  \centering
  \includegraphics[width=\textwidth]{encoder-reszletes.png}
  \caption{Encoder blokkvázlat részletesebben}
  \label{fig:encoder-reszletes}
\end{figure}


A(z) \ref{fig:encoder-reszletes}. ábrán látható egy részletesebb blokkvázlat az encoderről.

Az Analysis Polyphase Filterbankba megy a PCM input, itt az 1152 PCM minták mindegyike 32 egyenletesen eloszló frekvencia sávba lesz szétosztva. Mivel \( f_s = 44.1 kHz \) ezért a Nyquist frekvencia \( f_{Nyquist} = 22.05 kHz \). 
Minden sáv \[ \frac{22050}{32} = 689.0625 Hz \] széles lesz. Minden minta tartalmazhat 0-22.05 kHz-s komponenseket, ami a 32 sáv közül a megfelelőbe lesz szűrve.
Mivel minden mintát 32 subband-re osztunk, ezért az eredeti 1152-ből \( 1152*32 = 36 864 \) mintánk lesz. A folyamat végén viszont 32-vel decimálunk minden subband-et, így ismét 1152 mintánk lesz.

\newpage
\printbibliography

\end{document}
