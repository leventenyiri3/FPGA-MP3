\documentclass[12pt, letterpaper, a4paper]{article}
\usepackage{graphicx}
\usepackage{textcomp}
\usepackage[hungarian]{babel}
\usepackage[T1]{fontenc}
\usepackage[utf8]{inputenc}
\usepackage{caption}
\usepackage{subcaption}
\usepackage{csquotes}
\graphicspath{{./Pictures/}}
\usepackage[
    style=ieee,
  ]{biblatex}
\addbibresource{onlab.bib}

\usepackage{listings}
\usepackage[svgnames]{xcolor}
\usepackage{tikz}
\usetikzlibrary{shapes.geometric, arrows, calc}

\usepackage{float}
\usepackage{amsmath}
\usepackage{tabularx}

\definecolor{dkgreen}{rgb}{0,0.6,0}
\definecolor{gray}{rgb}{0.5,0.5,0.5}
\definecolor{mauve}{rgb}{0.58,0,0.82}


\lstset{frame=none,
  language=Bash,
  aboveskip=3mm,
  belowskip=3mm,
  showstringspaces=false,
  columns=flexible,
  basicstyle={\small\ttfamily},
  numbers=none,
  numberstyle=\tiny\color{gray},
  keywordstyle=\color{blue},
  commentstyle=\color{dkgreen},
  stringstyle=\color{mauve},
  breaklines=true,
  breakatwhitespace=true,
  tabsize=2
}

\lstdefinestyle{cstyle}{
  language=C,
  basicstyle=\footnotesize\ttfamily,  % smaller fixed-width font
  keywordstyle=\color{RoyalBlue},     % keywords like for, if, while
  commentstyle=\color{DarkGreen}\itshape,  % italic green comments
  stringstyle=\color{orange!80!black},     % orange strings
  identifierstyle=\color{black},      % variable/function names
  numberstyle=\tiny\color{gray},
  numbers=left,
  numbersep=8pt,
  frame=none,
  showspaces=false,
  showstringspaces=false,
  showtabs=false,
  tabsize=2,
  captionpos=b,
  breaklines=true,
  breakatwhitespace=true,
  aboveskip=0.5em,
  belowskip=0.5em,
  lineskip=-1pt,
  morekeywords={uint8_t, memcpy, memset}, % highlight C stdlib/uint types
  morekeywords=[2]{compare,merge,sort,media_filter_scalar},
  keywordstyle=[2]\color{purple}
}

\lstdefinestyle{pythonstyle}{
  language=Python,
  basicstyle=\ttfamily\footnotesize,
  keywordstyle=\color{RoyalBlue}\bfseries,
  commentstyle=\color{DarkGreen}\itshape,
  stringstyle=\color{orange!80!black},
  numberstyle=\tiny\color{gray},
  numbers=left,
  numbersep=8pt,
  frame=none,
  showstringspaces=false,
  tabsize=4,
  breaklines=true,
  breakatwhitespace=true,
  captionpos=b,
}

\tikzstyle{box} = [rectangle, rounded corners, minimum width=3cm, minimum height=0.5cm,text centered, draw=black, fill=orange!30]
\tikzstyle{arrow} = [thick,->,>=stealth]


\title{\textcolor{RoyalBlue}{\textbf{Audio tömörítő egység megvalósítása FPGA-val}}}
\author{\textcolor{RoyalBlue}{Nyiri Levente}}
\date{2025 Október}
\begin{document}
\begin{figure}
  \centering
  \includegraphics[width=\textwidth]{BME.png}
\end{figure}
\selectlanguage{hungarian}
\maketitle

\newpage

\section{Szabvány}

Tanulmányoztam az MPEG-1 Audio és az MPEG-2 Advanced Audio Coding (AAC) szabványokat, úgy döntöttem, hogy az MPEG-1 Audio-t fogom implementálni.

\section{LAME}

Kiindulásnak a LAME (Lame Aint an MP3 Encoder) nyílt forráskódú MP3 tömörítő szoftver forráskódját tanulmányoztam.
%TODO még írni kicsit a LAME-ről


\section{Encoder}

Az encoder blokkvázlata\cite{mp3-iso} az \ref{fig:encoder-diagram}. árbán látható.
Nem szabványosított az algoritmus, én a LAME source code-jából fogok kiindulni a tervezésnél.
A kimeneti bitsream-nek meg kell felelnie az International Standard-nak.

\begin{figure}[H]
  \centering
  \includegraphics[width=\textwidth]{encoder-diagram.png}
  \caption{Encoder blokkvázlat}
  \label{fig:encoder-diagram}
\end{figure}

\subsection{Mapping}


\subsection{Frame}

Egy MP3 file kisebb részegységekre van osztva, ezeket frame-eknek hívjuk. Minden frame 1152 audio mintát tartalmaz. Egy frame továbbá szét van választva 2 granule-ra, mindkettőre 576 minta jut.

Egy frame méretét byteban a(z) \ref{eq:frame-size}. ábra írja le.   

\begin{equation}
    \label{eq:frame-size}
    \text{frame size} = \frac{144 \cdot \text{bitrate}}{f_s} + \text{Padding}
\end{equation}
Egy frame felépítése a(z) \ref{tab:frame-header}. táblázatban \cite{raissi-mp3} látható.

\begin{table}[h]
\centering
\begin{tabular}{|c|c|c|c|c|}
  \hline
  \texttt{Header} & \texttt{CRC} & \texttt{Side\_Information} & \texttt{Main\_Data} & \texttt{Ancillary\_Data} \\
  \hline
\end{tabular}
\caption{Az MPEG frame fő mezői}
\label{tab:frame-header}
\end{table}

\subsubsection{Header}

A header tartalmazza a szinkronizációs biteket és egyéb információkat a frame-ről, felépítését mutatja a(z) \ref{fig:frame-header}. ábra \cite{raissi-mp3}.

\begin{figure}[H]
  \centering
  \includegraphics[width=0.2\textwidth]{frame-header.png}
  \caption{Frame header}
  \label{fig:frame-header}
\end{figure}

\begin{description}
  \item[Sync (12 bit)] Szerepe a szinkronizálás, mind a 12 bitnek 1-esnek kell lennie: \texttt{sync = 12'b1111\_1111\_1111}
  \item[ID (1 bit)] Az MPEG verziót határozza meg (MPEG-1 vagy MPEG-2)
  \item[Layer (2 bit)] Layer I, II vagy III 
  \item[Protection bit (1 bit)] Meghatározza, használunk-e CRC-t
  \item[Bitrate (4 bit)] A bitrate beállítása
  \item[Frequency (2 bit)] A mintavételi frekvencia meghatározása
  \item[Padding bit (1 bit)] Néhány frame-nek szüksége van rá, hogy a bitrate pontos legyen
  \item[Private bit (1 bit)] Applikáció-specifikus trigger
  \item[Mode (2 bit)] Csatornamód 
  \item[Mode extension (2 bit)] Csak joint stereo esetén használatos, további specifikáció
  \item[Copyright bit (1 bit)] Jelzi, hogy a tartalom szerzői jogi védelem alatt áll-e
  \item[Home bit (1 bit)] A frame az eredeti adathordozón található-e
  \item[Emphasis (2 bit)] A dekódernek szükséges-e de-emphasist alkalmaznia zajcsökkentés után
\end{description}

\subsubsection{Side information}

A side information további információt tartalmaz arra vonatkozóan, hogy hogyan kell dekódolni a frame-et. A felépítését a(z) \ref{tab:side-info}. táblázat mutatja.


\begin{table}[h]
\centering
\footnotesize
\begin{tabular}{|c|c|c|c|c|}
  \hline
  \texttt{main\_data\_begin} & \texttt{private\_bits} & \texttt{scfsi} & \texttt{Side\_info gr. 0} & \texttt{Side\_info gr. 1} \\
  \hline
\end{tabular}
\caption{A side information mezői}
\label{tab:side-info}
\end{table}

\newpage
\printbibliography

\end{document}
